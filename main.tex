
\def\mytitle{Boolean Expression to its simplest form using K-map}
\def\mykeywords{}
\def\myauthor{Arza Harsha}
\def\contact{harshaarza89@gmail.com}
\def\mymodule{Future Wireless Communication-(FWC22032)}
\documentclass[10pt, a4paper]{article}
\usepackage[a4paper,outer=1.5cm,inner=1.5cm,top=1.75cm,bottom=1.5cm]{geometry}
\twocolumn
\usepackage{graphicx}
\graphicspath{{./image/}}
\usepackage[colorlinks,linkcolor={black},citecolor={blue!80!black},urlcolor={blue!80!black}]{hyperref}
\usepackage[parfill]{parskip}
\usepackage{lmodern}
\renewcommand*\familydefault{\sfdefault}
\usepackage{watermark}
\usepackage{karnaugh-map}
\usepackage{lipsum}
\usepackage{xcolor}
\usepackage{listings}
\usepackage{float}
\usepackage{titlesec}
\usepackage{amsmath}
\usepackage{algorithm2e}

\titlespacing{\subsection}{0pt}{\parskip}{-3pt}
\titlespacing{\subsubsection}{0pt}{\parskip}{-\parskip}
\titlespacing{\paragraph}{0pt}{\parskip}{\parskip}
\newcommand{\figuremacro}[5]{
    \begin{figure}[#1]
        \centering
        \includegraphics[width=#5\columnwidth]{#2}
        \caption[#3]{\textbf{#3}#4}
        \label{fig:#2}
    \end{figure}
}

\lstset{
frame=single, 
breaklines=true,
columns=fullflexible
}
\thiswatermark{\centering \put(-22,-140.10){\includegraphics[scale=0.5]{IIT.jpg}} }
\title{\mytitle}
\author{\myauthor\hspace{1em}\\\contact\\IITH\hspace{0.5em}-\hspace{0.5em}\mymodule}
\date{}
\hypersetup{pdfauthor=\myauthor,pdftitle=\mytitle,pdfkeywords=\mykeywords}
\sloppy
\begin{document}
  \maketitle
\tableofcontents

\section{Introduction}
The external input signals A, B and C.( A B C , and are not available). The +5V power supply (logic 1) and the ground (logic 0) are also available. The output of the circuit is               
     X = A'B'+A'B'C'. 

        

\section{karnaugh-map}
   \begin{karnaugh-map}[4][2][1][$BC$][$A$]
        \minterms{0,2,3}
        \maxterms{1,4,5,6,7}
        \implicant{3}{2}
        \implicantedge{0}{0}{2}{2}
        \end{karnaugh-map} \\
         X = A'C'+A'B      
         
    


\section{Components}



\begin{table}[htbp]
 \begin{center}
    \begin{tabular}{|l|c|c|c|c|c|c} \hline \textbf{Component}
  & \textbf{value} & \textbf{quantity} \\
 \hline
Resistor & 220 ohm & 1 \\ \hline
Arduino & UNO & 1 \\ \hline
LED &  & 1 \\ \hline
Bread board &  & 1 \\ \hline
Jumper wires & M-M & 10\\ \hline
\end{tabular}   
\end{center}
\caption{\label{table:dummytable} }
\end{table}


\section{Truth table for given expression}
.
\begin{table}[htbp]
 \begin{center}
    \begin{tabular}{|l|c|c|c|c|c|c|c|c} \hline \textbf{A}
  & \textbf{B} & \textbf{C} & \textbf{X} \\
 \hline
        0&0&0&1 \\
        \hline
        0&0&1&0 \\
        \hline
        0&1&0&1 \\
        \hline
        0&1&1&1 \\
        \hline
        1&0&0&0 \\
        \hline
        1&0&1&0 \\
        \hline
        1&1&0&0 \\
        \hline
        1&1&1&0 \\
        \hline
\end{tabular}   
\end{center}
\caption{\label{table:dummytable} }
\end{table}







\section{Connections and results}



Also make connections to arduino UNO ,led and inputs based on table3. 

\begin{table}[htbp]
 \begin{center}
    \begin{tabular}{|l|c|c|c|c|c|c|c} \hline \textbf{Arduino UNO}
  & \textbf{2} & \textbf{3} & \textbf{4}& \textbf{8}& \textbf{gnd} \\
 \hline
Input&A&B&C&&\\ \hline
led&&&&+&- \\ \hline
\end{tabular}   
\end{center}
\caption{\label{table:dummytable} }
\end{table}


\begin{table}[htbp]
 \begin{center}
    \begin{tabular}{|l|c|c|c|c|c|c|c|c} \hline \textbf{Sample input}
  & \textbf{A} & \textbf{B} & \textbf{C}& \textbf{LED} \\
 \hline
1&0&0&0&ON\\ \hline
2&0&0&1&OFF \\ \hline
\end{tabular}   
\end{center}
\caption{\label{table:dummytable} }
\end{table}

\subsection{Code Link}
\vspace{5mm}
\begin{lstlisting}
https://github.com/Harsha-Arza/assignIDE/blob/main/codes/src/main.cpp
\end{lstlisting}
\end{document}


